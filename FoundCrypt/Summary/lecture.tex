\documentclass{scrartcl}
\usepackage[top=3cm, bottom=3cm, left=2cm,right=2cm]{geometry} 

\usepackage[english]{babel}
\usepackage[utf8]{inputenc}
\usepackage{mathtools}
\usepackage{esvect}
\usepackage{amssymb}
\usepackage{amsmath}

\usepackage{dcolumn}
\usepackage{booktabs}
\usepackage{tikz}
\usepackage{graphicx}
\usepackage{multicol}
\usepackage{float}
\usepackage{url}


\DeclareMathOperator*{\argmin}{argmin} % no space, limits underneath in displays
\DeclareMathOperator*{\argmax}{argmax} % no space, limits underneath in displays

\DeclarePairedDelimiter\abs{\lvert}{\rvert}%
\DeclarePairedDelimiter\norm{\lVert}{\rVert}%

\usepackage{titlesec}
\newcommand{\sectionbreak}{\clearpage}

\usepackage[parfill]{parskip}
\parskip = 4pt

\title{FoundCrypt - Summary}
\author{Sebastian Rietsch}

\begin{document}
\maketitle

\section{Introduction to Cryptography}
\subsection{Hash functions}
A \textbf{hash function} \(H\) maps strings of arbitrary input length to a string of a fixed length (e.g. 256 bits) (\(H: M \rightarrow R, \abs{M} >> \abs{R}\)).
It is \textbf{efficiently} computable for all inputs.

A \textbf{collision} occurs, if the hash function maps two distinct inputs \(x,y\) to the same output \(z\), i.e., \(H(x) = H(y)\) and \(x \neq y\).

A hash function \(H\) is \textbf{collision-resistant} if finding two values \(x,y\) such that \(H(x) = H(y)\) and \(x \neq y\) is not possible (in probabilistic polynomial-time) (every hash function has collisions, but finding them is assumed to be difficult).
\begin{itemize}
    \item
        Brute-force attack: Choose \(2^{256}+1\) distinct values, compute all hash values, output the first two colliding values
\end{itemize}

A hash function \(H\) is \textbf{hiding} if when a secret value \(r\) is chosen from a probability distribution that has high min-entropy, then given \(H(r || x)\) it is infeasible to find \(x\) (in probabilistic polynomial-time).

\subsection{Commitments}
\begin{itemize}
    \item
        \(com \leftarrow commit(k, m):\) The commitment algorithm takes as input a key \(k\), a message \(m\) and returns a commitment \(com\).
    \item
        \(\{0,1\} \leftarrow verify(com, k, m):\) The verification algorithm takes as input a commitment \(com\), a key \(k\), and a message \(m\). It outputs a bit.
\end{itemize}
Security properties:
\begin{itemize}
    \item
        \textbf{Hiding:} given \(com\), it is infeasible to find \(m\).
    \item
        \textbf{Binding:} for any key \(k\), it's infeasible to find \(x\neq y\) such that \(verify(commit(k,x), k, y)=1\).
\end{itemize}
Construction:
\begin{enumerate}
    \item
        Take key: \(r \leftarrow \{0,1\}^{256}\)
    \item
        Use hash function \(H\) to produce commitment: \(com \leftarrow H(r||m)\)
    \item
        Verification: Check that \(com = H(r||m)\)
\end{enumerate}

\subsection{Puzzle Friendliness}
A hash function \(H\) is \textbf{puzzle-friendly} if for every possible \(n\)-bit output value \(y\), if \(k\) is choosen from a distribution with high min-entropy, then it is infeasible to find \(x\) such that \(H(k||x) =y\) in time significantly less then \(2^n\).

Search puzzle:
\begin{itemize}
    \item
        \((H, id, Y)\) given, where \(id\) is the puzzle-ID, chosen from a high min-entropy distribution and \(Y\) is the target set (size determines the hardness)
    \item
        The value \(x\) is a solution if \(H(id||x) \in Y\)
\end{itemize}

A hash function \(H\) is \textbf{puzzle-friendly} if for every possible \(n\)-bit output value \(y\), if \(k\) is choosen from a distribution with high min-entropy, then it is infeasible to find \(x\) such that \(H(k||x) =y\) in time significantly less then \(2^n\).

\subsection{SHA-256 and the Merkle Damgard Transform}
TODO

\end{document}

