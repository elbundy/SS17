\documentclass[landscape, twocolumn]{article}
\usepackage{lscape}
\usepackage[german]{babel}
\usepackage[utf8]{inputenc}
\usepackage[left=1cm,right=1cm,top=1cm,bottom=1cm]{geometry} 


\title{Gilbert Ryle: Der Begriff des Geistes}
\author{Sebastian Rietsch \and Sophia Heilscher \and Sarah Bösendörfer}

\begin{document}
\maketitle

\section{Die offizielle Lehre}
\begin{itemize}
    \item
        Menschliche Wesen haben sowohl einen Körper als auch einen Geist und durchleben dadurch zwei parallele Lebensläufe
    \item
        Zwei verschiedenen Arten von Existenz/Sein: physisch vs. geistig
\end{itemize}

%\subsection*{Polare Gegensätze}
    \begin{center} \underline{\textbf{Körper}} \end{center}
        \begin{itemize}
            \item
                existiert im Raum, setzt sich aus Materie zusammen
            \item
                ist den mechanischen Kausalgesetzen unterworfen
                \begin{itemize}
                    \item
                        was einem Körper in einem Teil des Raumes zustößt, hängt mechanisch mit dem zusammen, was anderen Körpern in anderen Teilen des Raumes zustößt
                
                \end{itemize}
            \item
                Vorgänge können von äußeren Beobachtern wahrgenommen werden (öffentlich, äußerlich)
        \end{itemize}
     \begin{center} \underline{\textbf{Geist}} \end{center}
        \begin{itemize}
            \item
                existiert nicht im Raum, setzt sich aus Bewußtsein zusammen
            \item
                ist nicht den mechanischen Kausalgesetzen unterworfen
                \begin{itemize}
                    \item
                        keine direkte Kausalverknüpfung zwischen dem, was sich in einem Geist, und dem, was sich in einem anderen abspielt
                    \item
                        Geist eines Menschen kann nur über den Umweg über die physikalische Welt auf den eines anderen einwirken
                \end{itemize}
            \item
                Beobachter können nicht Zeugen dessen sein, was in jemandes Geist vorgeht (privat, innerlich)
            \item
                Introspektion ermöglicht jedem Menschen unmittelbares Wissen von den Vorgängen (Gedanken, Gefühle, Willensakte, Erinnerungen, Vorstellungen) in seiner eigenen Seele, frei von Täuschungen
                \begin{itemize}
                    \item
                        Erste Kritik
                
                \end{itemize}

            \item
                Eine Person kann bestenfalls Schlüsse aus dem beobachteten körperlichen Verhalten auf das Innenleben anderer schließen
                \begin{itemize}
                    \item
                        keine Garantie, dass gleiches körperliches Verhalten die selben geistigen Vorgänge als Ursache haben
                    \item
                        es existiert kein guter Grund, an die Existenz eines andern als ihres eigenen Geistes zu glauben
                    \item
                        Richtigkeit der sich auf das geistige und seelische Vermögen und Fähigkeiten anderer beziehen Worte unnütz
                \end{itemize}
\end{itemize}

\end{document}
