\documentclass[landscape, twocolumn]{scrartcl}
\usepackage{lscape}
\usepackage[german]{babel}
\usepackage[utf8]{inputenc}
\usepackage[left=1.5cm,right=1.5cm,top=1.5cm,bottom=1cm]{geometry} 
%\usepackage{atbegshi,picture}
%\usepackage{minibox}
\usepackage{lipsum}

\title{\vspace{-4ex}Gilbert Ryle: Der Begriff des Geistes\vspace{-0.3ex}}
\subtitle{Erstes Kapitel, Descartes' Mythos\vspace{-2ex}}
\author{Sophia Heilscher \and Sarah Bösendörfer \and Sebastian Rietsch}
\date{\vspace{-1.8ex}6. Juli 2017\vspace{-3ex}}

%\AtBeginShipout{\AtBeginShipoutUpperLeft{%
    %\put(\dimexpr\paperwidth-9.5cm\relax,-1cm){\minibox[r]{Tutorium zum Grundkurs Theoretischen Philosophie\\ Alma Thoma}}%
%}}



\begin{document}

\maketitle
%\noindent
\section{Die offizielle Lehre}
Menschliche Wesen haben sowohl einen Körper als auch einen Geist. Körper und Geist sind zusammengespannt, aber nach dem Tode des Körpers kann der Geist möglicherweise allein fortbestehen.

\noindent Eigenschaften von Körper und Geist sind polare Gegensätze.

%\subsection*{Polare Gegensätze}
\begin{center} \underline{\textbf{Körper}} \end{center}
\begin{itemize}
    \item
        existiert im Raum, setzt sich aus Materie zusammen, physikalische Welt
    \item
        ist den mechanischen Kausalgesetzen unterworfen
        \begin{itemize}
            \item
                was einem Körper in einem Teil des Raumes zustößt, hängt mechanisch mit dem zusammen, was anderen Körpern in anderen Teilen des Raumes zustößt
        \end{itemize}
    \item
        Vorgänge können von äußeren Beobachtern wahrgenommen werden (öffentlich, äußerlich)
\end{itemize}

\begin{center} \underline{\textbf{Geist}} \end{center}
\begin{itemize}
    \item
        existiert nicht im Raum, Welt der Geister
    \item
        ist nicht den mechanischen Kausalgesetzen unterworfen
        \begin{itemize}
            \item
                keine direkte Kausalverknüpfung zwischen dem, was sich in einem Geist, und dem, was sich in einem anderen abspielt
            \item
                Geist eines Menschen kann nur über den Umweg über die physikalische Welt auf den eines anderen einwirken
        \end{itemize}
    \item
        Beobachter können nicht Zeugen dessen sein, was in jemandes Geist vorgeht (privat, innerlich)
\end{itemize}
Jeder Mensch durchlebt also zwei parallele Lebensläufe/hat zwei Lebensgeschichten. Es gibt zwei verschiedene Arten von Existenzen: physisches und psychisches Sein.\begin{itemize}
    \item
        Grundlegende Frage: Wie können menschlicher Geist und Körper einander wechselseitig beeinflussen?
    \item
        Zusammenhänge zwischen den Vorfällen der privaten und jenen der öffentlichen Geschichte bleiben rätselhaft, da sie definitionsgemäß zu keiner der beiden Ereignisreihen gehören können
\end{itemize}

\smallskip

\noindent Bewußtsein und Introspektion ermöglicht jedem Menschen unmittelbares Wissen von den Vorgängen (Gedanken, Gefühle, Willensakte, Erinnerungen, Vorstellungen) in seiner eigenen Seele, ist frei von Täuschungen
\begin{itemize}
    \item
        \textit{Erste Kritik:} Leute werden von Trieben bewegt, deren Existenz sie anleugnen. Manche ihrer Gedanken sind ganz andere als die, die sie sich eingestehen. Manches was sie glauben zu wollen, wollen sie garnicht.

\end{itemize}

\noindent Eine Person kann bestenfalls aus dem beobachteten körperlichen Verhalten auf das Innenleben anderer schließen
\begin{itemize}
    \item
        keine Garantie, dass ähnliches körperliches Verhalten die selben geistigen Vorgänge als Ursache haben
    \item
        es existiert kein guter Grund, an die Existenz eines andern als ihres eigenen Geistes zu glauben
    %\item
        %fremde Geister können nicht entdeckt werden $\rightarrow$ absolute Einsamkeit des Seele
    \item
        anderen zugeschriebene Eigenschaften des Geistes, wie Intelligenz oder Charakter, können nicht auf Richtigkeit überprüft werden
\end{itemize}

\section{Die Absurdität der offiziellen Lehre}
\begin{itemize}
    \item
        Kategorienverwechslung:\\
        Beispiel: Der Ausländer und die Universität, steht für die Schwierigkeit, die aus deren Unfähigkeit erwuchsen, gewisse Wörter richtig zu verwenden
    \item
        Philosophisch interessante Kategorienverwechslung: werden von Leuten begangen, die mit Begriffen vertraut sind, aber aufgrund von abstrakten Gedankengängen diese manchmal der falschen Kategorie zuordnen\\
        Beispiel: Herr Müller und der Steuerzahler,\\
        steht für die Theorie vom menschlichen Doppelleben (Körper und Geist), ist eine Familie von Kategorienverwechslungen
    \item
        der menschliche Geist ist wie der menschliche Körper eine mannigfaltige und organisierte Einheit, nur dass er aus einem anderen Stoff und von anderer Art ist
    \item
        der menschliche Geist ist ein Feld von Ursachen und Wirkungen, nur diese sind nicht mechanischer Natur
    \item
        der Geist muss mit einer Parallelsprache beschrieben werden
\end{itemize}

\section{Der Ursprung der Kategorienverwechslung}
\begin{itemize}
    \item
        Wörter für geistiges Verhalten sind als Bezeichnung von nichtmechanischen Prozessen zu deuten
    \item
        Wirkung nichtmechanischer Ursachen sind geistige Vorgänge
    \item
        Antimechanisten, Geister werden als zusätzliche Zentren von Kausalvorgängen dargestellt
\end{itemize}

\subsubsection*{Offizielle Lehre}
Geist und Körper sind Gegensatzpaare $\rightarrow$ Geist definiert sich durch Verneinungen von Körperbeschreibungen, dadurch entspricht Geist derselben Logik wie der Körper

\subsubsection*{Ryle}
\begin{itemize}
    \item
        Körper als Motor, wird von einem andersartigen Motor im Inneren reguliert
    \item
        Innerer Motor: unsichtbar, unhörbar, hat weder Größe noch Gewicht, kann nicht zerlegt werden, die Gesetze denen er gehorcht, sind einem gewöhnlichen Ingenieur unbekannt, wie er den Körpermotor reguliert ist ebenfalls bekannt
\end{itemize}
Körper und Geist stammen aus verschiedenen Kategorien, können sich beeinflussen, aber funktionieren nicht nach derselben Logik

\subsubsection*{Annahme nach offizieller Lehre}
Die physikalische Welt ist ein deterministisches System, also muss die Geisterwelt ein deterministisches System sein.
\begin{itemize}
    \item
        Körper können die ihnen zugestoßenden Veränderungen nicht vermeiden, daher können die Geister nicht vermeiden, die ihnen vorzezeigte Laufbahn zu verfolgen
    \item
        Verantwortlichkeit, Wahl, Verdienst und Schuld sind daher unanwendbare Begriffe
\end{itemize}
Annahmen eines Außenstehenden über die Geisteszustände anderer Menschen sind sinnlos, da kein Zugriff auf die Psyche anderer möglich ist.\\
$\rightarrow$ Das Innenleben eines als schwachsinnig oder verrückt Erklärten könnte demnach ebenso rational sein, wie das eines jeden anderen

\subsubsection*{Ryle}
Wenn zwei Ausdrücke zur selben Kategorie gehören, ist es zulässig durch Konjunktion verbundene Sätze zu bilden.

\smallskip
\noindent Beispiel 1:

\textit{Ergibt Sinn:} \glqq Er hat einen linken und einen rechten Handschuh gekauft \grqq

\textit{Sinnlos:} \glqq  Er hat einen linken und einen rechten Handschuh und ein Paar Handschuhe gekauft \grqq

\medskip
\noindent Beispiel 2:

\textit{Ergibt Sinn:} \glqq Ein geisitger Vorgang hat sich abgespielt  \grqq

\textit{Ergibt Sinn:} \glqq Ein physischer Vorgang hat sich abgespielt \grqq\\
$\rightarrow$ Sinnlos wäre es beide Sätze mit Konjunktion 'und' zu verbinden
\end{document}
