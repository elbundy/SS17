\documentclass[a4paper]{article}

\usepackage[german]{babel}
\usepackage[utf8]{inputenc}
\usepackage{listings}
\usepackage{hyperref}

%\bibliographystyle{ieeetr}

\title{Essay Ryle}
\date{}
\author{Sebastian Rietsch}

\begin{document}
\maketitle
\section{Das K"orper-Geist-Problem}
Eines der größter Mysterien des menschlichen Daseins stellt die Komplexität des menschlichen, mentalen Innenlebens dar. Wie kann es sein, dass sich nach Jahrmillionen Lebewesen entwickelt haben, die die Fähigkeit besitzen, über komplexe Vorgänge nachzudenken, mentale Bilder zu erzeugen, mit Hilfe von Erfahrungen und Beobachtungen zukünftige Ereignisse vorherzusagen oder sich über seine eigene Existenz bewusst zu sein? Während sich andere Wissenschaftbereiche mit diesen Fragen beschäftigen, versucht die Philosophie des Geistes sich aus der Existenz der menschlichen Psyche ergebende Fragen zu beantworten. Eine zentrale Frage stellt hierbei das sogennante Körper-Geist-Problem dar, welches nach dem Verhältnis zwischen dem physischen Körpers auf der einen Seite, und den psychischen Vorgängen auf der anderen Seite, welche oft unter dem Oberbegriff des Geistes zusammengefasst werden, frägt. 

Eine der ersten Personen, die eine These zu diesem Problem aufstellte, und mit dieser auch weitesgehend auf Zustimmung stie"s, stellt Rene Descartes dar. 
%Der Kernpunkt seiner Lehre ist die strike Trennung von Körper und Geist, wobei auf die genauen Grundzüge erst im späteren Verlauf dieses Textes eingegangen wird.
Ein strikten Gegner dieser Lehrer ist Gilbert Ryle, welcher in \textit{\glqq Der Begriff des Geistes\grqq{}} Argumente liefert, wieso sie von Grund auf falsch sein muss. 

Im Folgenden wird die Sichtweise Ryles präsentiert. Genauer wird aufgezeigt, was nach Ryle die Grundannahmen der, wie er sie nennt, \textit{\glqq Offiziellen Lehre\grqq{}}, sind. Ihm zufolge beruht die Lehre auf einem sogenannten Kategorienfehler, dessen Bedeutung im Anschluss erkl"art wird.

\section{Descartes' Mythos}
Nach Ryle ist die zentrale Grundannahme der offiziellen Lehre, dass Körper und Geist getrennt voneinander existieren. Ein jeder Mensch besteht sowohl aus einem Körper, als auch einem Geist. Diese sind während des Lebens zusammengesponnen, wobei sich der Geist unter Umständen nach dem Tod vom Körper trennen kann, und in Folge dessen alleine fortbesteht und seine Funktionen ausübt.  

Die Eigenschaften dieser beiden \glqq Entitäten\grqq{} werden hierbei sehr gegensätzlich charakterisiert. Während der Körper in Raum und Zeit existiert, existiert der Geist nur in der Zeit. Einerseits ist der Körper ist den mechanischen Kausalgesetzen unterworfen, was bedeutet, dass das, was einem Körper in einem Teil des Raumes zustö"st kausal mit dem zusammenhängt, was mit anderen Körpern in anderen Teilen des Raumes geschieht. Im Gegensatz dazu unterliegt der Geist keinen mechanischen Kausalgesetzen, sondern einem nicht-phyischem, parameschnischem Kausalgesetz, welches ein komplementäres Gegenstück des mechanischen Kausalgesetzes darstellt (Ryle merkt hierbei an, dass ein Geist nur auf andere Geister einwirken kann, indem er den \glqq Umweg\grqq{} über die physische Welt auf sich nimmt und sozusagen den an ihn angebunden Körper für diesen Zweck verwendet). Zu guter Letzt sind körperliche Vorgänge von au"sen sichtbar. Ein jeder Mensch kann die Körper anderer Menschen äu"serlich beobachten, k"orperliche Vorg"ange sind also öffentlich. Geistige Vorgänge hingegen sind nicht öffentlich. Andere Beobachter können nicht Zeuge davon sein, was in jemandes Geistes vorgeht. Die Vorgänge des Geistes sind also privat, nur der Geist selbst kann von ihnen direkte Kenntnis haben.

Ryle nennt als weitere Annahme der Lehre des Descartes die Fähigkeit der täuschungsfreie Introspektion. Unter dieser wird die Fähigkeit verstanden, einen klaren Blick auf die eigenen inneren Vorgänge (Gedanken, Gefühle, Willensakte, Erinnerungen, Vorstellungen) werfen zu können. Diese Annahme stellt einen klaren Angriffspunkt für Ryle dar, da es offensichtlich ist, dass ein jeder Mensch zu Zeiten nicht weiss, was genau in seinem Inneren vor sich geht. Dabei beruft er sich beispielsweise auf die wissenschaftlichen Erkentnisse von Sigmund Freud.

Seiner Ansicht nach ist dieses Bild des Geistes mit einigen Problemen verbunden. Das grö"ste Problem stellt die Tatsache dar, dass die Art und Weise, mit der Geist und Körper aufeinander wechselseitig einwirken nicht erklärt werden kann. Da der kausale Zusammenhang zwischen Körper und Geist definitionsgemä"s weder Teil der öffentlichen, physischen, noch des privaten, geistigen Bereiches sein kann, bleibt dieser mysteriös und rätselhaft. Ein anderes Problem ist die Tatsache, dass man nach diesen Annahmen nie wissen kann, was genau im Inneren eines anderen Menschen vorgehen. Die einzige Möglichkeit auf die inneren Vorgänge eines anderen zu schlie"sen besteht darin, dessen körperliche Vorgänge zu beobachten und mit Erfahrungen des eigenen Körpers zu vergleichen. Jedoch wird hier eine weiter Annahme getroffen, nämlich dass ähnliche körperliche Vorgänge der anderen Person ähnliche geistige Vorgänge als Ursache haben. Sich auf fremende Geister beziehende Eingeschaftszuschreibungen wie Intelligenz oder Charakter verlieren in Folge dessen vollkommen an Bedeutung. Zwei weitere, recht triste Schlussfolgerungen die nach Ryle hieraus gezogen werden müssten sind, dass es keinen Grund gibt and die existenz fremder Geister zu glauben und die totale Einsamkeit der Seele.

\section{Kategorienverwechslung}
Auf Grund dieser sich ergebenden Probleme ist Ryle der Meinung, dass die offizielle Lehre fehlerhaft sein muss und auf einem einzig großen Irrtum beruht. Bei diesem Irrtum handelt es sich nach ihm um eine sogenannte Kategorieverwechslung.

Um den Begriff der Kategorieverwechslung verständlicher zu machen nennt Ryle mehrere Beispiele. Man stelle sich einen Ausländer vor, der zum ersten Mal nach Cambridge oder Oxford kommt und dem eine Reihe von Colleges, Bibliotheken, Sportplätzen, Museen, Laboratorien und Verwaltungsgebäude gezeigt werden. Nach einiger Zeit fragt dieser, wo dann die Universität sei. In Folge dessen muss man ihm erklären, dass die Universität keine weiter Institution sei, sondern einfach die Art und Weise, in der alles was er schon gesehen hat organisiert ist. Der Fehler des Ausländers lag also darin anzunehmen, die Einrichtungen der Universität in dieselben Kategorie einzureihen wie die Universität selbst. 

Ein weiteres Beispiel stellt ein Politikstudent dar, der die Hauptunterschiede zwischen der englischen, franz"osischen und amerikanischen Verfassung und auch die Unterschiede und Zusammenh"ange zwischen dem Kabinett, dem Parlament, den verschiedenen Ministerien, der Richterschaft und der englischen Staatskirche gelernt hat. Wenn er nun nach den Zusammenhängen zwischen der englischen Staatskirche, dem Innenminister und der Verfassung Englands gefragt wird, gerät er jedoch in Verlegenheit, dann während die Kirche und das Ministerium Institutionen sind, ist die Verfassung nicht im selben Sinne eine weitere Institution. Daher können inter-institutionelle Beziehungen, deren Bestehen zwischen Kirche und Innenministerium man behaupten oder leugnen kann, keine Beziehungen sein, deren Bestehen zwischen Kirche und Innenministerium einerseits und der englischen Verfassung andererseits man behaupten oder leugnen könnte. Auch in diesem Fall wurde der Fehler begangen, Ausdrücke unterschiedlicher Kategorien so zu verwedenden, als würden sie der selben Kategorie angehören.

%Es sollte anhand des letzten Abschnittes ungefähr klargeworden sein, was Ryle unter unter einer Kategorienverwechslung verstehen. Im Folgenden wird dieser Begriff genau denifiniert. 

Nach Ryle begeht also derjenige einen Kategoriefehler, der einen Ausdruck \(a\) so behandelt, als gehöre er zur Kategorie \(A\), während er zur Kategorie \(B\) gehört. Er nennt dabei zwei Merkamle, anhand derer festgestellt werden kann, ob mehrere sprachliche Ausdrücke derselben Kategorie angehören oder nicht.

Einerseits wird die Kategorie eines Ausdrucks durch die Klasse der sprachlogisch richtigen Verwendungen dieses Ausdrucks bestimmt. In anderen Worten bedeutet dies, dass zwei Ausdrücke \(a\) und \(b\) zu derselben Kategorie gehören, genau dann wenn \(a\) in allen Kontexten, in denen die Verwendung von \(a\) sinnvoll ist, durch \(b\) ersetzen kann, ohne dass Unsinn entsteht. Als Beispiel nehme man die Ausdrücke Primzahl und dreieckig. Während der Satz \glqq{}Die Zahl 7 ist eine Primzahl\grqq{} sinnvoll ist, ist der Satz \glqq{}Die Zahl 7 ist dreieckig\grqq{} unsinnig. Die beiden Ausdrücke können also nicht der selben sprachlichen Kategorie angehören. Auf die selbe Weise kann man hingegen aber auch feststellen, dass die Ausdrücke \glqq{}Primzahl\grqq{} und \glqq{}Natürliche Zahl\grqq{} der selben Kategorie angehören, da der Satz \glqq{}Die Zahl 7 ist eine Natürliche Zahl\grqq{} sinnvoll ist.

Als zweites Indiz nennt Ryle, dass zwei Ausdrücke derselben Kategorie angehören, wenn man sie in Und-Sätzen miteinander verbinden kann. Als unsinniges Beispiel wird hier der Satz \glqq{}Sie kam heim in einer Flut von Tränen und in einer Sänfte.\grqq{} Es ist erkennbar, dass \glqq{}in einer Flut von Tränen\grqq{} und \glqq{}in einer Sänfte\grqq{} nicht derselben Kategorie angehören können. Dass anhand dieses Merkmals Ausdrücke unterschiedlicher, beziehungsweise gemeinsamer Kategorien erkannt werden können sollte auch durch die am Anfang dieses Abschnittes genannten Beispiele klar geworden sein.
\end{document}
