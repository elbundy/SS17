\documentclass[a4paper]{article}

\usepackage[german]{babel}
\usepackage[utf8]{inputenc}
\usepackage{listings}
\usepackage{hyperref}

%\bibliographystyle{ieeetr}

\title{Philosophische Leitfragen}
\date{}
\author{Sebastian Rietsch}


\begin{document}
\maketitle

\setcounter{section}{2}
\setcounter{subsection}{0}
\subsection{1963 bringt der Philosoph Edmund Gettier eine fundamentale Kritik gegen die Standardanalyse des Wissens vor. Welche Kritik übt Gettier an der Standardanalyse?}
Vor der Publikation Gettiers Aufsatzes wurde weitverbreitet angenommen, Wissen ließe sich als gerechtfertigte wahre Meinung analysieren. Die Standardanalyse des Wissens definiert Wissen genauer wie folgt:

\begin{center}
    Ein Subjekt $S$ weiß, dass $p$, genau dann wenn:
    \begin{enumerate}
        \item
            $p$ wahr ist
        \item
            $S$ glaubt, dass $p$
        \item
            $S$ darin gerechtfertigt ist zu glauben, dass $p$
    \end{enumerate}
\end{center}
Gettier zeigt mit Hilfe eines Beispiels gerechtfertigte, wahre Meinung, die dennoch kein Wissen ist. Er versucht zu verdeutlichen, dass die Bedigungen nicht hinreichend sind \cite{gettier}. 

Zwei Arbeitssuchende names Smith und Jones haben sich bei einer Firma um diesselbe Stelle beworben. Smith hat glaubhaft vom Personalchef erfahren, dass sich die Firma letzten Endes für seinen Konkurrenten Jones entscheiden wird. Er hat außerdem gesehen, dass Jones zehn Münzen in seiner Hosentasche mit sich herumträgt. Damit hat er gute Gründe für die folgenden beiden Überzeugungen:
\begin{enumerate}
    \item
        Jones ist derjenige, der die Stelle bekommt, und hat zehn Münzen in der Hosentasche
    \item
        Derjenige, der die Stelle bekommt, hat zehn Münzen in seiner Hosentasche
\end{enumerate}
In der zweiten Überzeugung ist er gerechtfertigt, da sie eine logische Implikation der ersten darstellt, in welcher er durch die vom Personalchef erhaltenen Informationen und seinen eigenen Beobachtungen gerechtfertigt ist.

Nun ereignen sich zwei Dinge, von denen Smith nichts weiss. Erstens hat Smith selbst ebenfalls zehn Münzen in seiner Hosentasche und zweitens bekommt er und nicht Jones am Ende die Stelle. Demnach ist die zweite Überzeugung Smiths, dass derjenige die Stelle bekommt zehn Münzen in der Hosentasche hat wahr und gerechtfertigt. Dennoch kann man ihm in diesem Fall kein Wissen zuschreiben, da sich die Überzeugung nur durch puren Zufall als wahr erweist. Smith hat gewissermaßen Glück im Unglück, da er von einer gerechtfertigten Überzeugung, die unglücklicherweise falsch ist, eine wahre Überzeugung logisch deduktiv ableitet.

\subsection{Wie könnte man versuchen, die klassische Analyse zu modifizieren, um das von Gettier entworfene Szenario auszuschließen}
In Gettiers Beispiel kommt die gerechtfertigte, wahre Meinung durch einen Schluss aus falschen Prämissen zustande. Eine Möglichkeit das entworfene Szenario auszuschließen wäre deshalb zu fordern, dass man in einer Überzeugung nur gerechtfertigt sein kann, wenn diese aus wahren Prämissen geschlossen wurde \cite{gettier1}. Genauer:
\begin{center}
    Wenn $S$ korrekterweise $q$ aus $p$ ableitet, dann ist $S$ nur dann gerechtfertigt, $q$ zu glauben, wenn $p$ wahr ist
\end{center}
%Feldman (1974, An Alleged Defect in Gettier-Counterexamples) hat jedoch gezeigt, dass auch diese Definition nicht hinreichend ist.
Ebenso könnte man die Möglichkeit eliminieren, dass sich eine gerechtfertigte Überzeugung aus falsch heraustellen kann. In dem oben aufgezeigten Beispiel stellte sich heraus, dass die vom dem Personalchef erhaltene Information sich als falsch herausstellte. Eine weiter Forderung könnte deshalb lauten, dass Wissen sich auf unfehlbare, absolut sichere Evidenzen stützen muss. Das Problem an dieser Einschränkung ist jedoch, dass es dadurch nahezu unmöglich wird etwas zu wissen, da sie empirisch erhobene Evidenzen ausschließt \cite{gettier2}.

Eine andere Antwort auf das Problem stellt die Forderung nach Kausalität dar, da sich alle Gettierbeispiele darin ähneln, dass es in ihnen keinen adäquaten Zusammenhang zwischen der Überzeugung und der Tatsache gibt, die diese Überzeugung wahr macht. Die Wissensdefinition muss deshalb um folgendes Kriterium ergänzt werden:

\begin{center}
    Die Überzeugung von $S$, dass $p$, wurde durch die Tatsache verursacht, die $p$ wahr macht \cite{gettier1}
\end{center}

%Ein Grund, weshalb in Gettiers Beispiel  , dass die 
%Es könnte ebenso gefordert werden, dass die Evidenzen, aus denen die Überzeugungen entstehen
\setcounter{section}{7}
\setcounter{subsection}{0}
\subsection{Wird die Bedeutung eines sprachlichen Ausdrucks allein durch die geistigen Zustände der Sprecher festgelegt?}
Für die Beantwortung dieser Frage stellt Putnam \cite{putnam} ein Beispiel auf. Man stelle sich vor es existiere innerhalb der Milchstraße ein Zwillingsplanet der Erde, welcher Zwerde genannt wird und der Erde bis auf den folgenden Unterschied exakt gleicht: Während das, was Erdlinge als Wasser bezeichnen $H_2 0$ ist, ist das Wasser der Zwerdlinge ein anderes, kompliziertes Molekül, welches der Einfachheit halber als $XYZ$ bezeichnet wird. Es wird außerdem angenommen, dass die beiden Flüssigkeiten ohne chemische Analyseverfahren nicht unterschieden werden können.

Würde nun ein Raumschiff der Erde die Zwerde besuchen, so würden die Erdlinge festellen, dass 'Wasser' auf der Zwerde nicht $H_2 O$, sondern $XYZ$ bedeutet und umgekehrt. Das Wort 'Wasser' hat in diesem Fall also zwei verschiedene Bedeutungen, je nachdem in welchem Kontext es verwendet wird.

Man drehe nun die Zeit zurück in das Jahr 1750. Sowohl auf der Erde, wie auch auf der Zwerde ist die Chemie als Wissenschaft noch nicht weitgenug entwickelt um Wasser auf seine Zusammensetzung zu untersuchen. Sei Oskar$_1$ ein zu dieser Zeit auf der Erde lebender Mensch und Oskar$_2$ sein zwerdisches Gegenstück, welcher ihm in Aussehen, Gedanken, Gefühlen, inneren Monologen etc. exakt gleicht. Man stellt nun fest, dass das was Oskar$_1$ unter Wasser auffässt das selbe ist wie das, was Erdlinge in der Gegenwart unter Wasser verstehen, nämlich $H_2 O$. Das gleiche gilt für Oskar$_2$, welcher $XYZ$ als Wasser bezeichnet. Oskar$_1$ und Oskar$_2$ fassen den Ausdruck 'Wasser' also unterschiedlich auf, obwohl sie sich im selben geistigen Zustand befinden.

Die Bedeutung eines sprachlichen Ausdrucks kann also \textbf{nicht} alleine durch die geistigen Zustände der Sprecher festgelegt werden.

\bibliography{lit}
\bibliographystyle{ieeetr}
\end{document}
