\documentclass[a4paper]{article}

\usepackage[english]{babel}
\usepackage[utf8]{inputenc}
\usepackage{mathtools}
\usepackage{esvect}


\DeclareMathOperator*{\argmin}{argmin} % no space, limits underneath in displays
\DeclareMathOperator*{\argmax}{argmax} % no space, limits underneath in displays

\title{Computer Vision - Summary}
\author{Sebastian Rietsch}

\begin{document}
\maketitle
\section*{4.1.1 Feature detectors (Harris corner detector)}
\begin{itemize}
    \item
        Simplest possible matching criterion for comparing two image patches: (weighted) summed square difference
        $$E_{WSSD}(u) = \sum_i w(x_i) [I_1 (x_i + u) - I_0(x_i)]^2$$
        where $I_0$ and $I_1$ are the two images being compared, $u = (u,v)$ is the displacement vector
    \item
        Idea: compare an image patch against itself, which is known as an \textit{auto-correlation function}
        $$E_{AC}(\Delta u) = \sum_i w(x_i) [I_0(x_i + \Delta u) - I_0(x_i)]^2$$
    \item
        Using Taylor Series expansion $I_0(x_i + \Delta u) \approx I_0(x_i) + \Delta I_0(x_i) \cdot \Delta u$ we can approximate the auto-correlation function as
        $$E_{AC}(\Delta u) = \dots = \sum_i w(x_i) [\Delta I_0(x_i) \cdot \Delta u]^2 = \Delta u^T A \Delta u$$
        where $A = w * \begin{pmatrix} I_x^2 & I_x I_y \\ I_x I_y & I_y^2  \end{pmatrix}$ ($I_x, I_y$ are gradients in $x$ and $y$ direction)
    \item
        Compute $I_x, I_y$ with $[-2, -1, 0, 1, 2]$ filter or derivatives of Gaussians. Use Gaussian as weighting kernel $w$.
    \item
        Auto-correlation matrix contains useful information. Two ways to extract information:
        \begin{enumerate}
            \item
                Both eigenvalues big $\Rightarrow$ corner, one eigenvalue big $\Rightarrow$ edge

            \item
                Compute $det(A) - \alpha trace(A)^2 = \lambda_0 \lambda_1 - \alpha (\lambda_0 + \lambda_1)^2$. Threshold and non-maximum supression.
        \end{enumerate}
    \item
        Algorithm:
        \begin{enumerate}
            \item
                Compute the horizontal and vertical derivatives of the image $I_x$ and $I_y$ by convolving the original image with derivatives of Gaussians
            \item
                Compute the three images corresponding to the outer products of these gradients ($I_x^2, I_y^2, I_x I_y$)
            \item
                Convolve each of these images with a larger Gaussian
            \item
                Compute a scalar interest measure using one of the formulas discussed above
            \item
                Find local maxima above a certain threshold and report them as detected feature point locations.
                
        \end{enumerate}

\end{itemize}


\section*{Multi View Geometry: Chapter 9: Epipolar Geometry and the Fundamental Matrix}
\subsection*{Epipolar geometry}
\begin{itemize}
    \item
        \textit{Epipolar plane} $\pi$: formed by camera centers $c, c'$ and space point $X$ (or by camera centers and one image point $x$)
    \item
        Image points $x$ and $x'$ lie on epipolar plane (they are coplanar)
    \item
        Suppose we know $x$ and $x'$ is unknown: $x'$ must lie on the line of intersection $l'$ of $\pi$ with the second image plane
    \item
        $l'$ is the so-called \textit{epipolar line}
    \item
        intersection of the line joining the camera centers is called \textit{epipole}
\end{itemize}

\subsection*{Fundamental matrix $F$}
\begin{itemize}
    \item
        $3 \times 3$ matrix
    \item
        Rank 2
    \item
        2 points $x$ and $x'$ $\Rightarrow$ $x'^T F x = 0$
    \item
        Epipolar line is the projection in the second image of the ray from the point $x$ through the camera center $c$ of the first camera $\Rightarrow$ there is a map $x \rightarrow l'$
\end{itemize}
\subsubsection*{Algebraic derivation}
\begin{itemize}
    \item
        Given: two camera projection matrices $P, P'$
    \item
        The ray back-projected from $x$ by $P$ is obtained by solving $PX = x$. Solution: $X(\lambda) = P^{+} x + \lambda C$, where $P^{+}$ is the pseudo-inverse of $P$ and $C$ its null-vector, namely the camera center defined by PC = 0
    \item
        Two points on the ray are $P^{+}x$ $(\lambda = 0)$ and the first camera center $C$ $(\lambda = \infty)$
    \item
        Second camera images these points at $P'P^{+}x$ and $P'C$
    \item
        The epipolar line is the line joining these two projected points, namely $l' = (P'C) \times (P'P^{+}x)$
    \item
        $P'C$ is the epipole in the second image and may be denoted as $e'$. Thus $l' = [e']_{\times} (P'P^{+})x = Fx$
    \item
        $F = [e']_{\times} (P'P^{+})$
\end{itemize}

Suppose we have to images acquired by cameras with non-coincident centres, then the \textbf{fundamental matrix} $F$ is the unique $3 \times 3$ rank 2 homogeneous matrix which satisfies $x'Fx = 0$ for all corresponding points $x \leftrightarrow x'$. $(l' = Fx)$

\subsection*{The essential matrix}
\begin{itemize}
    \item
        Camera matrix $P = K[R|t]$
    \item
        Coordinate: $x = PX$
    \item
        Normalized coordinate: $\hat{x} = [R|t] X$ (image of the point $X$ with respect to a camera $[R|t]$ having the identity matrix $I$ as calibration matrix
    \item
        Normalized camera matrix: $K^{-1} P = [R|t]$
    \item
        ... now
    \item
        Consider a pair of normalized camera matrices $P = [I | 0]$ and $P' = [R|t]$
    \item
        The fundamental matrix corresponding to the pair of normalized cameras is customarily called the \textit{essential matrix}
    \item
        $E = [t]_{\times} R = R[R^{T} t]_{\times}$
    \item
        $\hat{x}'^T E \hat{x} = 0$ in terms of normalized image coordinates for corresponding points $x \leftrightarrow x'$
    \item
        Subsituting $\hat{x}$ and $\hat{x'}$ gives $\hat{x}'^T K'^{-T} E K^{-1} \hat{x} = 0$, which then leads to $E = K'^T F K$
\end{itemize}

\subsubsection*{Extraction of cameras from the essential matrix}
\begin{itemize}
    \item
        Assume the first camera matrix is $P = [I|0]$
    \item
        Compute SVD of $E$, which is $U diag(1,1,0) V^T$
    \item
        $W = \begin{bmatrix}
            0 & -1 & 0 \\ 1 & 0 & 0 \\ 0 & 0 & 1
        \end{bmatrix}$
        $Z = \begin{bmatrix}
            0 & 1 & 0 \\ -1 & 0 & 0 \\ 0 & 0 & 0
        \end{bmatrix}$
    \item
        There are two possible factorizations $E = SR$, namely $S = UZU^T$ and $R = UWV^T$ or $R = U W^T V^T$
    \item
        There are four possible choices for the second camera matrix $P'$
    \item
        $P' = [UWV^T | +u_3]$ or $[UWV^T| -u_3]$ or $[UW^TV^T| + u_3]$ or $[UW^TV^T| -u_3]$ where $u_3$ is the last column of $U$
\end{itemize}
\end{document}
