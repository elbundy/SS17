\documentclass{scrartcl}
\usepackage[top=3cm, bottom=3cm, left=2cm,right=2cm]{geometry} 

\usepackage[english]{babel}
\usepackage[utf8]{inputenc}
\usepackage{mathtools}
\usepackage{esvect}
\usepackage{amssymb}
\usepackage{amsmath}

\usepackage{dcolumn}
\usepackage{booktabs}
\usepackage{tikz}
\usepackage{graphicx}
\usepackage{multicol}
\usepackage{float}
\usepackage{url}


\DeclareMathOperator*{\argmin}{argmin} % no space, limits underneath in displays
\DeclareMathOperator*{\argmax}{argmax} % no space, limits underneath in displays

\DeclarePairedDelimiter\abs{\lvert}{\rvert}%
\DeclarePairedDelimiter\norm{\lVert}{\rVert}%

\usepackage{titlesec}
\newcommand{\sectionbreak}{\clearpage}

\usepackage[parfill]{parskip}
\parskip = 4pt

\title{Künstliche Intelligenz 2 - Summary}
\author{Sebastian Rietsch}

\begin{document}
\maketitle

\section{Probablistic Reasoning, Part I: Basics}
\subsection{Introduction}
\textbf{Sources of Uncertainty} in Decision-Making:
\begin{itemize}
    \item
        Non-deterministic actions
    \item
        Partial observability with unreliable sensors
    \item
        Uncertainty about the domain behavior
\end{itemize}

What is \textbf{probabilistic reasoning}?
\begin{itemize}
    \item
        Deducing probabilities from knowledge about \textit{other} probabilities
    \item
        Determines probabilities that are difficult to assess, based on probabilities that are (relatively) easy to assess (\textit{to asses:} the process of estimating a probability \(P\) using statistics)
\end{itemize}

\textbf{Rational Agents}:
\begin{itemize}
    \item
        We have a choice of \textbf{actions}
    \item
        These can lead to different solutions with different probabilities
    \item
        The actions have different costs
    \item
        The results have different utilities
    \item
        A rational agent chooses the action with the \textbf{maximum expected utility}
\end{itemize}

\subsection{Unconditional Probabilities}
A \textbf{probability theory} is an assertion language for talking about possible worlds and an inference method for quantifying the degree of belief in such assertions (Example: we roll two dice with six sides, then we have 36 possible worlds: \((1,1), (2,1), \dots, (6,6)\)).

A \textbf{probability model} \(\langle \Omega, P \rangle\) consists of a set \(\Omega\) of possible worlds called the sample space and a probability function \(P: \Omega \rightarrow \mathbb{R}\), such that \(0 \leq P(\omega) \leq 1\) for all \(\omega \in \Omega\) and \(\sum_{\omega \in \Omega} P(\omega) = 1\). We restrict ourselves to a discrete, countable sample space.

\textbf{Random variables}:
\begin{itemize}
    \item
        A random variable is a variable quantity whose value depends on possible outcomes of unknown variables and processes we do not understand
    \item
        Given a random variable \(X\), \(P(X = x)\) denotes the prior probability, or unconditional probability, that \(X\) has value \(x\)
    \item
        We will refer to the fact \(X = x\) as an event or outcome
    \item
        For Boolean variable \(Name\), we write name for \(Name = \top\) and \(\lnot name\) for \(Name = \bot\)
\end{itemize}

The \textbf{probability distribution} for a random variable \(X\), written \(P(X)\), is the vector of probabilities for the (ordered) domain of \(X\).

Given a subset \(Z \subseteq \{X_1, \dots, X_n\}\) of random variables, an event is an assignment of values to the variables in \(Z\). The \textbf{joint probability distribution}, written \(P(Z)\), lists the probabilities of all events.

An \textbf{atomic event} is an assignment of values to all variables.

A \textbf{proposition} describes a set of multiple atomic events (\textbf{Example:} we roll two dice and are interested in the cases where they add up to 11). The probability associated with a proposition is defined by the sum of probabilities of the worlds in which it holds: For any proposition \(\phi, P(\phi) = \sum_{\omega \in \phi} P(\phi)\).

\textbf{Kolmogorov axioms:}
\begin{enumerate}
    \item
        The probability of an event is a non-negative real number: \[P(E) \in \mathbb{R}, P(E) \geq 0, \forall E \in F\] where \(F\) is the event space.
    \item
        The probability that at least one of the elementary events in the entire sample space will occure is 1: \(P(\Omega) = 1\).
    \item
        Any countable sequence of disjoint sets (mutually exclusive events) \(E_1, E_2, \dots \) satisfies \[P(\bigcup_{i=1}^{\infty} (E_i)) = \sum_{i=1}^{\infty} P(E_i).\]
\end{enumerate}
From this follows \(P(A \lor B) = P(A) + P(B) - P(A \land B)\).

\subsection{Conditional Probabilities}
In the presence of additional information, we can no loger use the unconditional (\textbf{prior}) probabilities (because probabilies model our belief, thus they depend on our knowledge).

Given propositions \(A\) and \(B\), \(P(a|b)\) denotes the \textbf{conditional probability} of \(a\) given that all we know is \(b\). The conditional, or posterior probability of \(a\) given \(b\) is defined as:
\[P(a|b) = \frac{P(a \land b)}{P(b)}\]

The \textbf{conditional probability distribution} of \(P(X|Y)\) is the table of all conditional probabilies of values of \(X\) given values of \(Y\).

\subsection{Independence}
Problem with full joint probility distributions: Given \(n\) random variables with \(k\) values each, the joint pdf containts \(k^n\) probabilities. This motivates the use of conditional probabilities while exploiting the \textbf{(conditional) independence} property.

Events \(a\) and \(b\) are \textbf{independent} if \(P(a \land b) = P(a) \cdot P(b)\), from which follows that \(P(a | b) = P(a)\) and vice versa.

In case of independence, the joint probability distribution of multiple variables can be reconstructed from smaller, statistically independent joint probability distributions.

\subsection{Basic Probabilistic Reasoning Methods}
\begin{itemize}
    \item
        \textbf{Product Rule:} \(P(a \land b) = P(a | b) P(b)\) 
    \item
        \textbf{Chain Rule:} \(P(X_1, \dots, X_n) = P(X_n | X_{n-1}, \dots X_1) \cdot \dots \cdot P(X_2 | X_1) \cdot P(X_1)\) (This works for any ordering of the variables)
    \item
        \textbf{Marginalization:} Given sets \(X\) and \(Y\) of random variables, we have:
        \[P(X) = \sum_{y \in Y} P(X, y)\]
    \item
        \textbf{Normalization:} Given a vector \(\langle w_1, \dots, w_k \rangle\) of numbers in \([0,1]\) where \(\sum_{i=1}^k w_i \leq 1\), the normalization constant \(\alpha\) is \(\alpha \langle w_1, \dots, w_k \rangle = 1/\sum_{i=1}^k w_i\)
        \begin{itemize}
            \item
                Given a random variable \(X\) and an event \(e\), we have \(P(X|e) = \alpha P(X, e)\) with \(\alpha = 1/P(e)\)
            \item
                Normalization + Marginalization: Given "query variable" \(X\), "observed event" \(e\), and "hidden variables" set \(Y\):
                \[P(X|e) = \alpha P(X, e) = \alpha \sum_{y \in Y} P(X, e, y)\]
        \end{itemize}
\end{itemize}

\subsection{Bayes' Rule}
Given propositions \(A\) and \(B\) where \(P(a) \neq 0\) and  \(P(b) \neq 0\), we have:
\[P(a|b) = \frac{P(b|a) P(a)}{P(b)}\]

\subsection{Conditional Independence}
Given sets of random variables \(Z_1, Z_2 \text{ and } Z\), we say \(Z_1\) and \(Z_2\) are \textbf{conditionally independent given \(Z\)} if:
\[P(Z_1, Z_2 | Z) = P(Z_1 | Z) \cdot P(Z_2 | Z)\]

If \(Z_1\) and \(Z_2\) are conditionally independent given \(Z\), the \(P(Z_1 | Z_2, Z) = P(Z_1 | Z)\).

\subsection{Summary}
\begin{itemize}
    \item
        Uncertainty is unavoidable in many environsments, namely whenever agents do not have perfect knowledge
    \item
        Probabilities express the degree of belief of an agent, given its knowledge, into an event
    \item
        Conditional probabilities express the likelihood of an event given observed evidence
    \item
        Assessing a probability means to use statistics to approximate the likelihood of an event
    \item
        Bayes' rule allows us to derive, from probabilities that are easy to assess, probabilities that arn't easy to assess
    \item
        Given mulitple evidence, we can exploit conditional independencie
\end{itemize}

\section{Probablistic Reasoning, Part II: Bayesian Networks}
\end{document}

